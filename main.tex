\documentclass[openright]{normas-utf-tex}

\special{papersize=210mm,297mm}
\usepackage[hidelinks,plainpages=true,bookmarks=true,bookmarksdepth=6,breaklinks,pdfstartview=FitH,
    pdftitle={Os efeitos das técnicas de respiração controlada no manejo da dor em parturientes durante o parto vaginal},
    pdfauthor={Laiz Camili Ferreira},
    pdfsubject={Projeto de pesquisa em fisioterapia obstétrica},
    pdfkeywords={Respiração controlada; Parto vaginal; Fisioterapia obstétrica}]{hyperref}
\usepackage[hyphenbreaks]{breakurl}
\urlstyle{same}
\usepackage[alf,abnt-emphasize=bf,bibjustif,recuo=0cm,abnt-etal-cite=2,abnt-etal-list=0,abnt-thesis-year=both]{abntcite}
\usepackage[brazil]{babel}
\usepackage[utf8]{inputenc}
\usepackage{amsmath,amsfonts,amssymb}
\usepackage{graphicx}
\usepackage[T1]{fontenc}
\usepackage{lmodern}
\usepackage[final]{pdfpages}
\usepackage{float}
\usepackage{pifont}

\newcommand{\cmark}{\ding{51}}
\newcommand{\xmark}{\ding{55}}

\instituicao{Centro Universitário de Pato Branco}
\instituicaoSigla{UNIDEP}
\programa{Curso de Fisioterapia}
\programaSigla{FISIOTERAPIA}
\area{Fisioterapia Obstétrica}

\documento{Projeto de Pesquisa}
\documentoingles{Research Project}
\nivel{Graduação}
\curso{Fisioterapia}
\titulacao{Bacharel}
\autorizacao{(\cmark) Não autorizo a disponibilização de endereço de correio eletrônico para contato.\\
(\xmark) Autorizo a disponibilização do seguinte correio eletrônico para contato: laiz.ferreira@email.com}
\licencaCCBY{}
\licencaCCBYNC{}
\licencaCCBYNCND{X}
\licencaCCBYNCSA{}
\licencaCCBYND{}
\licencaCCBYSA{}

\titulo{Os efeitos das técnicas de respiração controlada no manejo da dor em parturientes durante o parto vaginal}
\title{The effects of controlled breathing techniques on pain management in women during vaginal childbirth}

\autor{Laiz Camili Ferreira}
\cita{FERREIRA, Laiz Camili}

\palavraschave{Respiração controlada; Trabalho de parto; Fisioterapia obstétrica}
\keywords{Controlled breathing; Labor; Obstetric physiotherapy}

\comentario{Projeto de pesquisa apresentado como requisito parcial para obtenção do título de Bacharel em Fisioterapia do Curso de Fisioterapia do Centro Universitário de Pato Branco (UNIDEP).}

\orientador[Orientadora:]{Prof.ª Me. Thais Feihrmann}

\local{Pato Branco}
\data{2025}

\tolerance=1
\emergencystretch=\maxdimen
\hyphenpenalty=10000
\hbadness=10000
\sloppy

\begin{document}
\pdfstringdefDisableCommands{%
        \let\MakeUppercase\relax
}
\capa
\folhaderosto

\begin{dedicatoria}
\vspace*{\fill}
A Deus e à minha família, pela força concedida em cada etapa desta jornada.\\[1ex]
\vspace*{\fill}
\end{dedicatoria}

\begin{agradecimentos}
Agradeço à minha orientadora, Prof.ª Me. Thais Feihrmann, pela orientação generosa, e às parturientes e doulas que se dispuseram a contribuir com este estudo, tornando-o possível.
\end{agradecimentos}

\begin{epigrafe}
\begin{flushright}
\emph{``O amor tudo desculpa, tudo crê, tudo espera, tudo suporta.'' (1Cor 13,7)}
\end{flushright}
\end{epigrafe}

\begin{resumo}
Este projeto de pesquisa propõe analisar como técnicas de respiração controlada podem auxiliar no manejo da dor de parturientes durante o trabalho de parto vaginal. A intervenção será desenvolvida com uma gestante atendida no contexto da fisioterapia obstétrica, com sessões domiciliares de orientação e acompanhamento presencial e remoto durante o parto. Pretende-se compreender a percepção de alívio da dor, a capacidade de aplicação das técnicas ao longo das fases do parto e a contribuição do método para um nascimento humanizado. A investigação, de caráter qualitativo e descritivo, buscará evidenciar as repercussões das estratégias respiratórias tanto para a parturiente quanto para o fortalecimento da atuação fisioterapêutica.
\end{resumo}

\begin{abstract}
This research project proposes to analyse how controlled breathing techniques may support pain management for women in labor during vaginal childbirth. The intervention will be conducted with a pregnant participant assisted in the context of obstetric physiotherapy, including home-based guidance sessions and on-site and remote follow-up during labor. The study aims to understand the perceived pain relief, the participant's ability to apply the techniques throughout labor, and the contribution of the method to a humanized birth. This qualitative descriptive investigation seeks to highlight the impact of respiratory strategies on both the pregnant woman and the strengthening of physiotherapy practice.
\end{abstract}

\listadefiguras
\listadetabelas

\sumario

\chapter{Introdução}
\label{chap:introducao}

A gestação é um período de profundas transformações físicas e emocionais na vida da mulher, no qual o parto assume um papel fundamental, sendo marcado por desafios intensos e emoções diversas. As dores que acompanham o trabalho de parto são respostas fisiológicas complexas, que resultam principalmente da interação de fatores físicos e psicológicos. Dessa forma, é essencial a busca por estratégias que promovam o bem-estar da parturiente, aliviando a dor e auxiliando no enfrentamento desse momento \cite[p.~54]{almeida2005}.

A escolha ou indicação do tipo de parto exerce influência significativa sobre a vivência da mulher durante o trabalho de parto e o nascimento. O parto vaginal, seja espontâneo ou assistido, é reconhecido como a via fisiológica de nascimento, associada a menores riscos de complicações para a mãe e o recém-nascido em comparação com a cesariana, conforme aponta a Organização Mundial da Saúde \cite{oms2018}.

A fisioterapia desempenha um papel crucial no preparo físico e emocional da mulher para o parto. Entre as suas abordagens, destacam-se o fortalecimento e a mobilização do assoalho pélvico, além de técnicas de respiração controlada, que são fundamentais no alívio da dor durante o trabalho de parto. A respiração controlada, como técnica não farmacológica, contribui significativamente para a redução da percepção da dor, promove o relaxamento muscular e melhora a oxigenação tanto da mãe quanto do bebê, facilitando o processo de parto \cite{cortes2015}.

Esse cuidado integral, que respeita a autonomia e o protagonismo da mulher, é fundamental não apenas no contexto físico e emocional, mas também na perspectiva cristã, pois o matrimônio é compreendido como uma vocação sagrada, ordenada à comunhão conjugal e à geração da vida. De acordo com o Catecismo da Igreja Católica \cite[p.~432]{catecismo2000}, ``os filhos são o dom mais excelente do matrimônio e constituem a plenitude desse vínculo, conferindo aos esposos uma missão de responsabilidade e amor''.

Dessa maneira, a busca por práticas que respeitem o corpo da mulher e promovam um parto mais humanizado reflete uma vivência de amor e cuidado, alinhada a valores humanitários e de fé. Nesse cenário, a fisioterapia desempenha papel fundamental ao oferecer recursos que aliviam a dor, reduzem a ansiedade e promovem o bem-estar da parturiente por meio da respiração controlada, tornando possível um nascimento mais respeitoso, saudável e verdadeiramente alinhado às necessidades físicas e emocionais de todos os envolvidos.

\section{Problema de Pesquisa}
\label{sec:problema}

De que maneira as técnicas de respiração controlada podem contribuir para o alívio da dor em parturientes durante o parto vaginal?

\section{Objetivos}
\label{sec:objetivos}

\subsection{Objetivo Geral}
\label{subsec:objetivo-geral}

Analisar a eficácia das técnicas de respiração controlada como estratégia não farmacológica para o alívio da dor em parturientes durante o trabalho de parto vaginal.

\subsection{Objetivos Específicos}
\label{subsec:objetivos-especificos}

\begin{itemize}
    \item Identificar a ação das técnicas de respiração controlada aplicadas durante o trabalho de parto.
    \item Avaliar qualitativamente a percepção do alívio da dor da parturiente acompanhada que utilizou as técnicas de respiração controlada durante as contrações.
    \item Analisar a capacidade da parturiente em colocar em prática as técnicas propostas até o período expulsivo.
    \item Discutir, a partir de um estudo de caso, a contribuição da respiração controlada como método de suporte ao parto humanizado.
\end{itemize}

\chapter{Justificativa}
\label{chap:justificativa}

A dor durante o parto é uma experiência intensa que pode gerar ansiedade e desconforto significativos nas parturientes. A Associação Internacional para o Estudo da Dor define a dor como uma experiência sensorial e emocional desagradável associada ou semelhante àquela associada a um dano real ou potencial aos tecidos, envolvendo dimensões físicas e emocionais do sofrimento \cite{raja2020}. Nesse contexto, a busca por métodos não farmacológicos de alívio da dor, como as técnicas de respiração e relaxamento, tem ganhado destaque, visando proporcionar um parto mais humanizado.

Estudos demonstram que a prática de técnicas de respiração e relaxamento influencia positivamente a experiência da parturiente. \citeonline{almeida2005} observaram que, embora a intensidade da dor aumentasse com a progressão do trabalho de parto, a utilização dessas técnicas promoveu a manutenção de níveis mais baixos de ansiedade por um período prolongado. Além disso, tais métodos podem reduzir o medo e o uso de analgésicos e anestésicos, favorecendo um parto mais satisfatório e natural.

A atuação fisioterapêutica no trabalho de parto revela-se essencial ao oferecer suporte no manejo da dor e da ansiedade, além de incentivar o protagonismo da mulher por meio de recursos como exercícios respiratórios, mobilidade pélvica, estimulação elétrica nervosa transcutânea e hidroterapia, que auxiliam na evolução do trabalho de parto, promovendo maior conforto e bem-estar da parturiente \cite{sousa2021}.

Assim, com base nas vivências adquiridas durante a graduação em Fisioterapia, especialmente no contato com gestantes, percebe-se a relevância de aprofundar o conhecimento em métodos não farmacológicos de alívio da dor, reafirmando o papel da fisioterapia como promotora de uma assistência obstétrica mais acolhedora, segura e centrada na mulher.

\chapter{Metodologia}
\label{chap:metodologia}

Este estudo trata-se de uma pesquisa descritiva, com abordagem qualitativa, do tipo relato de caso. O objetivo será analisar a eficácia da aplicação de técnicas de respiração controlada no manejo da dor durante o trabalho de parto vaginal, por meio da experiência de uma parturiente acompanhada pela pesquisadora no contexto da fisioterapia obstétrica.

A pesquisa será realizada com uma gestante residente na cidade de Pato Branco, selecionada entre pacientes que estão na fila de espera ou em atendimento na Clínica Escola de Fisioterapia do Centro Universitário de Pato Branco (UNIDEP). Após a identificação das gestantes, será feito contato com as possíveis participantes. Aquelas que demonstrarem interesse e atenderem aos critérios de inclusão estabelecidos serão convidadas individualmente a participar do estudo. A primeira gestante que aceitar o convite e estiver conforme os critérios será selecionada como participante do relato de caso.

Os critérios de inclusão serão: ser primigesta, ter idade entre 25 e 35 anos, evoluir para trabalho de parto normal e de baixo risco, não ser portadora de nenhuma patologia associada a complicações obstétricas e não ter participado de grupos de preparo fisioterapêutico para o parto. Serão excluídos casos de sofrimento fetal agudo, patologias obstétricas com indicação de cesariana, necessidade de parto fórceps ou uso de analgesia.

A parturiente, durante o trabalho de parto, receberá assistência individualizada de uma doula, com orientação e estímulo à utilização de técnicas de respiração e relaxamento durante o processo de parturição.

A pesquisa seguirá um protocolo de coleta de dados dividido em quatro etapas: seleção das possíveis gestantes para participação no relato de caso; ensino das técnicas de respiração na residência da parturiente; aplicação das técnicas durante o trabalho de parto; e, por fim, o relato da parturiente sobre a aplicação das técnicas e seus possíveis resultados.

Após a aprovação do Comitê de Ética em Pesquisa do Centro Universitário de Pato Branco (UNIDEP), será dado início à aplicação das técnicas de respiração na residência da gestante selecionada, durante o segundo semestre do ano de 2025, bem como no hospital escolhido por ela, sendo este o Hospital Filantrópico Policlínica ou o Hospital Filantrópico São Lucas, ambos localizados na cidade de Pato Branco, Paraná.

A intervenção ocorrerá de forma individual, sendo realizadas três sessões de treinamento respiratório na casa da gestante durante um mês, com duração média de 40 minutos cada. Essas sessões terão como objetivo orientar a parturiente na prática de técnicas como respiração abdominal profunda, respiração torácica lenta e respiração de pressão, adaptadas conforme as fases do trabalho de parto. Além dos encontros presenciais, a participante será incentivada a praticar os exercícios em casa, com apoio e monitoramento remoto via WhatsApp. O acompanhamento da aplicação das técnicas também será feito durante o trabalho de parto, em ambiente hospitalar, com supervisão da pesquisadora e de uma fisioterapeuta, garantindo a segurança da paciente.

A técnica aplicada neste estudo foi baseada no protocolo descrito por \citeonline{almeida2005}, o qual propõe estratégias de respiração e relaxamento como forma de favorecer a experiência do trabalho de parto. Embora este seja o principal referencial metodológico adotado, é importante destacar que o artigo utilizado também se apoia em autores clássicos como Maldonado \citeyear{maldonado1991} e Lucas \citeyear{lucas1983}, que exploram os aspectos psicológicos da gestação, do parto e do puerpério, bem como a facilitação psicológica do parto. Esses referenciais oferecem um embasamento teórico complementar que reforça a importância das técnicas utilizadas na assistência à parturiente.

\section{Cronograma das Atividades}
\label{sec:cronograma}

\begin{table}[H]
    \centering
    \caption{Cronograma de atividades previsto para 2025}
    \label{tab:cronograma}
    \begin{tabular}{|p{6cm}|p{7.5cm}|}
        \hline
        \textbf{Atividades a serem executadas} & \textbf{Cronograma de execução} \\
        \hline
        Elaboração do projeto & Maio \\
        Entrega final do projeto & Maio \\
        Contato com a UNIDEP & Maio \\
        Obtenção da autorização da UNIDEP & Maio \\
        Apreciação/aprovação da Plataforma Brasil & Maio a Julho \\
        Coleta de dados & Julho \\
        Tabulação e interpretação dos dados & Setembro a Outubro \\
        Análise e discussão dos resultados & Setembro a Novembro \\
        Finalização do artigo & Dezembro \\
        Defesa do trabalho & Dezembro \\
        Entrega da versão final & Dezembro \\
        \hline
    \end{tabular}
    \vspace{0.5em}
    \fonte{Elaboração da autora.}
\end{table}

\noindent\textit{Nota: o cronograma será executado caso o projeto seja aprovado pelo Sistema CEP/CONEP. A coleta de dados iniciará somente após a autorização do Comitê de Ética em Pesquisa do Centro Universitário de Pato Branco.}

\section{Orçamento}
\label{sec:orcamento}

O desenvolvimento deste Trabalho de Conclusão de Curso não demandará recursos financeiros significativos, sendo viabilizado, em sua maior parte, por meio de recursos próprios e ferramentas gratuitas. A Tabela~\ref{tab:orcamento} apresenta a estimativa dos custos envolvidos.

\begin{table}[H]
    \centering
    \caption{Estimativa de custos para elaboração da pesquisa}
    \label{tab:orcamento}
    \begin{tabular}{|p{5cm}|c|c|c|}
        \hline
        \textbf{Itens} & \textbf{Custo unitário (R\$)} & \textbf{Quantidade} & \textbf{Total (R\$)} \\
        \hline
        Impressões & 0,80 & 65 folhas & 52,00 \\
        Deslocamento para ensinar as técnicas & 15,00 & 3 sessões & 45,00 \\
        Internet & 20,00 & 4 meses & 80,00 \\
        \hline
        \multicolumn{3}{|r|}{\textbf{Custo total estimado}} & \textbf{177,00} \\
        \hline
    \end{tabular}
    \vspace{0.5em}
    \fonte{Elaboração da autora.}
\end{table}

\section{Aspectos Éticos}
\label{sec:etica}

Esta pesquisa será submetida à apreciação do Comitê de Ética em Pesquisa, em conformidade com a Resolução nº 466/2012 do Conselho Nacional de Saúde, que trata das diretrizes e normas regulamentadoras de pesquisas envolvendo seres humanos. A participante será devidamente informada sobre os objetivos do estudo, e os riscos envolvidos são considerados mínimos, relacionando-se à possibilidade de exposição indevida de dados pessoais. Esses riscos serão mitigados por meio de codificação, sigilo das informações e assinatura do Termo de Consentimento Livre e Esclarecido, bem como do termo de uso de imagem, assegurando o anonimato em registros audiovisuais.

Entre os benefícios esperados, destacam-se a contribuição para o fortalecimento da fisioterapia obstétrica, a valorização de sua atuação no contexto do parto humanizado e a importância das técnicas não farmacológicas, como a respiração controlada, na promoção do bem-estar da parturiente. Ademais, o estudo visa ressaltar os impactos positivos das práticas fisioterapêuticas na experiência do trabalho de parto, especialmente quando realizadas em um contexto acolhedor e respeitoso às necessidades físicas, emocionais e espirituais da mulher.

\bibliography{main}

\end{document}

\pdfminorversion=7
\documentclass[openright]{tex/estilos/normas-utf-tex}

\makeatletter
\def\input@path{{tex/estilos/}{}}
\makeatother

\special{papersize=210mm,297mm}
\PassOptionsToPackage{hyphens}{url}
\usepackage[hidelinks,plainpages=true,bookmarks=true,bookmarksdepth=6,pdfstartview=FitH,
    pdftitle={Os efeitos das técnicas de respiração controlada no manejo da dor em parturientes durante o parto vaginal},
    pdfauthor={Laiz Camili Ferreira Olenik},
    pdfsubject={Trabalho de Conclusão de Curso em fisioterapia obstétrica},
    pdfkeywords={Respiração controlada; Parto vaginal; Fisioterapia obstétrica}]{hyperref}
\urlstyle{same}
\providecommand{\burl}[1]{\url{#1}}
\usepackage[alf,abnt-emphasize=bf,bibjustif,recuo=0cm,abnt-etal-cite=2,abnt-etal-list=0,abnt-thesis-year=both]{tex/estilos/abntcite}
\usepackage[brazil]{babel}
\usepackage[utf8]{inputenc}
\usepackage{amsmath,amsfonts,amssymb}
\usepackage{graphicx}
\usepackage[T1]{fontenc}
\usepackage{lmodern}
\usepackage[final]{pdfpages}
\usepackage{float}
\usepackage{pifont}

\newcommand{\cmark}{\ding{51}}
\newcommand{\xmark}{\ding{55}}

\instituicao{Centro Universitário de Pato Branco}
\instituicaoSigla{UNIDEP}
\programa{Curso de Fisioterapia}
\programaSigla{FISIOTERAPIA}
\area{Fisioterapia Obstétrica}

\documento{Trabalho de Conclusão de Curso}
\documentoingles{Undergraduate Thesis}
\nivel{Graduação}
\curso{Fisioterapia}
\titulacao{Bacharel}
\autorizacao{(\cmark) Não autorizo a disponibilização de endereço de correio eletrônico para contato.\\
(\xmark) Autorizo a disponibilização do seguinte correio eletrônico para contato: laiz.ferreira@email.com}
\licencaCCBY{}
\licencaCCBYNC{}
\licencaCCBYNCND{X}
\licencaCCBYNCSA{}
\licencaCCBYND{}
\licencaCCBYSA{}

\titulo{Os efeitos das técnicas de respiração controlada no manejo da dor em parturientes durante o parto vaginal}
\title{The effects of controlled breathing techniques on pain management in women during vaginal childbirth}

\autor{Laiz Camili Ferreira}
\cita{FERREIRA, Laiz Camili}

\palavraschave{Respiração controlada; Trabalho de parto; Fisioterapia obstétrica}
\keywords{Controlled breathing; Labor; Obstetric physiotherapy}

\comentario{Trabalho de Conclusão de Curso apresentado como requisito parcial para obtenção do título de Bacharel em Fisioterapia do Curso de Fisioterapia do Centro Universitário de Pato Branco (UNIDEP).}

\orientador[Orientadora:]{Prof.ª Me. Thais Feihrmann}

\local{Pato Branco}
\data{2025}

\tolerance=1
\emergencystretch=\maxdimen
\hyphenpenalty=10000
\hbadness=10000
\sloppy

\begin{document}
\pdfstringdefDisableCommands{%
        \let\MakeUppercase\relax
}
\hypersetup{pageanchor=false}
\capa
\folhaderosto
\pagenumbering{roman}

\begin{dedicatoria}
\vspace*{\fill}
Dedica-se este trabalho a Deus e à família da autora, em reconhecimento à força concedida em cada etapa desta jornada.\\[1ex]
\vspace*{\fill}
\end{dedicatoria}

\begin{agradecimentos}
A autora agradece à orientadora, Prof.ª Me. Thais Feihrmann, pela orientação generosa, bem como às parturientes e doulas que se dispuseram a contribuir com este estudo, tornando-o possível.
\end{agradecimentos}

\begin{epigrafe}
\begin{flushright}
\emph{``O amor tudo desculpa, tudo crê, tudo espera, tudo suporta.'' (1Cor 13,7)}
\end{flushright}
\end{epigrafe}

\begin{resumo}
Este Trabalho de Conclusão de Curso analisou como técnicas de respiração controlada auxiliaram no manejo da dor de uma parturiente durante o trabalho de parto vaginal. A intervenção foi desenvolvida com uma gestante atendida no contexto da fisioterapia obstétrica, contemplando sessões domiciliares de orientação e acompanhamento presencial e remoto ao longo do parto. Buscou-se compreender a percepção de alívio da dor, a capacidade de aplicação das técnicas nas diferentes fases do parto e a contribuição do método para um nascimento humanizado. A investigação, de caráter qualitativo e descritivo, evidenciou as repercussões das estratégias respiratórias tanto para a parturiente quanto para o fortalecimento da atuação fisioterapêutica, cujos dados e interpretações encontram-se organizados nos capítulos de Resultados e Discussão.
\end{resumo}

\begin{abstract}
This undergraduate thesis analysed how controlled breathing techniques supported pain management for a woman in labor during vaginal childbirth. The intervention was conducted with a pregnant participant assisted within the context of obstetric physiotherapy, including home-based guidance sessions and on-site and remote follow-up throughout labor. The study sought to understand perceived pain relief, the participant's ability to apply the techniques during different stages of labor, and the contribution of the method to a humanized birth. This qualitative descriptive investigation highlighted the impact of respiratory strategies on both the participant and the strengthening of physiotherapy practice, with findings discussed in the chapters dedicated to Results and Discussion.
\end{abstract}

\listadefiguras
\listadetabelas

\sumario

\clearpage
\hypersetup{pageanchor=true}
\pagenumbering{arabic}

\chapter{Introdução}
\label{chap:introducao}

A gestação é um período de profundas transformações físicas e emocionais na vida da mulher, no qual o parto assume um papel fundamental, sendo marcado por desafios intensos e emoções diversas. As dores que acompanham o trabalho de parto são respostas fisiológicas complexas, que resultam principalmente da interação de fatores físicos e psicológicos. Dessa forma, é essencial a busca por estratégias que promovam o bem-estar da parturiente, aliviando a dor e auxiliando no enfrentamento desse momento \cite[p.~54]{almeida2005}.

A escolha ou indicação do tipo de parto exerce influência significativa sobre a vivência da mulher durante o trabalho de parto e o nascimento. O parto vaginal, seja espontâneo ou assistido, é reconhecido como a via fisiológica de nascimento, associada a menores riscos de complicações para a mãe e o recém-nascido em comparação com a cesariana, conforme aponta a Organização Mundial da Saúde \cite{oms2018}.

A fisioterapia desempenha um papel crucial no preparo físico e emocional da mulher para o parto. Entre as suas abordagens, destacam-se o fortalecimento e a mobilização do assoalho pélvico, além de técnicas de respiração controlada, que são fundamentais no alívio da dor durante o trabalho de parto. A respiração controlada, como técnica não farmacológica, contribui significativamente para a redução da percepção da dor, promove o relaxamento muscular e melhora a oxigenação tanto da mãe quanto do bebê, facilitando o processo de parto \cite{cortes2015}.

A abordagem humanizada do parto, ao reconhecer a gestante como sujeito integral, demanda que a equipe multiprofissional considere dimensões físicas, psicoemocionais e espirituais, fomentando redes de apoio que respeitem sua dignidade e liberdade decisória. Tal perspectiva converge com o Magistério da Igreja Católica, que compreende o matrimônio e a maternidade como vocações voltadas ao cuidado amoroso da vida nascente, atribuindo aos pais a responsabilidade de resguardar o bem-estar da mãe e do filho \cite[p.~431-432]{catecismo2000}. Essa visão encontra ressonância nas Sagradas Escrituras, que celebram o corpo como templo do Espírito Santo e reconhecem cada nova vida como dom divino, conforme 1Cor 6,19 e Sl 139,13-14 na edição Ave-Maria da Bíblia \cite{biblia2018}. Ao incorporar tais valores éticos às condutas baseadas em evidências, a fisioterapia obstétrica fortalece vínculos de confiança, favorece o protagonismo feminino e amplia a compreensão da respiração controlada como recurso de cuidado integral.

Dessa maneira, a busca por práticas que respeitem o corpo da mulher e promovam um parto mais humanizado reflete um compromisso com a dignidade humana e com a fé professada pela família, oferecendo suporte consistente para decisões informadas. Nesse cenário, a fisioterapia desempenha papel fundamental ao oferecer recursos que aliviam a dor, reduzem a ansiedade e promovem o bem-estar da parturiente por meio da respiração controlada, tornando possível um nascimento mais respeitoso, saudável e alinhado às necessidades físicas, emocionais e espirituais de todos os envolvidos.

O estudo relatado neste trabalho investigou a aplicação dessas técnicas em uma parturiente acompanhada no contexto da fisioterapia obstétrica, buscando compreender os efeitos percebidos, a execução das estratégias respiratórias e as contribuições para uma experiência de parto humanizada. A seguir, apresentam-se o problema de pesquisa, os objetivos delineados e os procedimentos metodológicos adotados para alcançar tais propósitos.

\section{Problema de Pesquisa}
\label{sec:problema}

O estudo investigou de que maneira as técnicas de respiração controlada contribuíram para o alívio da dor em parturientes durante o parto vaginal.

\section{Objetivos}
\label{sec:objetivos}

\subsection{Objetivo Geral}
\label{subsec:objetivo-geral}

O objetivo geral foi analisar a eficácia das técnicas de respiração controlada como estratégia não farmacológica para o alívio da dor em parturientes durante o trabalho de parto vaginal.

\subsection{Objetivos Específicos}
\label{subsec:objetivos-especificos}

Foram estabelecidos os seguintes objetivos específicos:

\begin{itemize}
    \item Identificar a ação das técnicas de respiração controlada aplicadas durante o trabalho de parto.
    \item Avaliar qualitativamente a percepção do alívio da dor da parturiente acompanhada que utilizou as técnicas de respiração controlada durante as contrações.
    \item Analisar a capacidade da parturiente em colocar em prática as técnicas propostas até o período expulsivo.
    \item Discutir, a partir de um estudo de caso, a contribuição da respiração controlada como método de suporte ao parto humanizado.
\end{itemize}

\chapter{Justificativa}
\label{chap:justificativa}

A dor durante o parto é uma experiência intensa que pode gerar ansiedade e desconforto significativos nas parturientes. A Associação Internacional para o Estudo da Dor define a dor como uma experiência sensorial e emocional desagradável associada ou semelhante àquela associada a um dano real ou potencial aos tecidos, envolvendo dimensões físicas e emocionais do sofrimento \cite{raja2020}. Nesse contexto, a busca por métodos não farmacológicos de alívio da dor, como as técnicas de respiração e relaxamento, tem ganhado destaque, visando proporcionar um parto mais humanizado.

Estudos demonstram que a prática de técnicas de respiração e relaxamento influencia positivamente a experiência da parturiente. \citeonline{almeida2005} observaram que, embora a intensidade da dor aumentasse com a progressão do trabalho de parto, a utilização dessas técnicas promoveu a manutenção de níveis mais baixos de ansiedade por um período prolongado. Além disso, tais métodos podem reduzir o medo e o uso de analgésicos e anestésicos, favorecendo um parto mais satisfatório e natural.

A atuação fisioterapêutica no trabalho de parto revela-se essencial ao oferecer suporte no manejo da dor e da ansiedade, além de incentivar o protagonismo da mulher por meio de recursos como exercícios respiratórios, mobilidade pélvica, estimulação elétrica nervosa transcutânea e hidroterapia, que auxiliam na evolução do trabalho de parto, promovendo maior conforto e bem-estar da parturiente \cite{sousa2021}.

Assim, com base nas vivências adquiridas durante a graduação em Fisioterapia, especialmente no contato com gestantes, percebe-se a relevância de aprofundar o conhecimento em métodos não farmacológicos de alívio da dor, reafirmando o papel da fisioterapia como promotora de uma assistência obstétrica mais acolhedora, segura e centrada na mulher.

\chapter{Metodologia}
\label{chap:metodologia}

Este estudo tratou-se de uma pesquisa descritiva, com abordagem qualitativa, do tipo relato de caso. O objetivo consistiu em analisar a eficácia da aplicação de técnicas de respiração controlada no manejo da dor durante o trabalho de parto vaginal, a partir da experiência de uma parturiente acompanhada no contexto da fisioterapia obstétrica.

A pesquisa foi realizada com uma gestante residente na cidade de Pato Branco, selecionada entre pacientes que estavam na fila de espera ou em atendimento na Clínica Escola de Fisioterapia do Centro Universitário de Pato Branco (UNIDEP). Após a identificação das possíveis participantes, realizou-se contato individual para apresentação do estudo. As gestantes que demonstraram interesse e atenderam aos critérios de inclusão foram convidadas a participar. A primeira gestante que aceitou o convite e se enquadrou em todos os requisitos foi selecionada como participante do relato de caso.

Os critérios de inclusão abrangeram ser primigesta, ter idade entre 25 e 35 anos, evoluir para trabalho de parto normal e de baixo risco, não apresentar patologias associadas a complicações obstétricas e não ter participado de grupos de preparo fisioterapêutico para o parto. Foram excluídos casos de sofrimento fetal agudo, patologias obstétricas com indicação de cesariana, necessidade de parto fórceps ou uso de analgesia.

Durante o trabalho de parto, a parturiente recebeu assistência individualizada de uma doula, com orientação e estímulo à utilização de técnicas de respiração e relaxamento ao longo do processo de parturição.

A pesquisa seguiu um protocolo de coleta de dados dividido em quatro etapas: seleção das possíveis gestantes para participação no relato de caso; ensino das técnicas de respiração na residência da parturiente; aplicação das técnicas durante o trabalho de parto; e relato da parturiente sobre a aplicação das técnicas e seus resultados percebidos.

Após a aprovação do Comitê de Ética em Pesquisa do Centro Universitário de Pato Branco (UNIDEP), deu-se início à aplicação das técnicas de respiração na residência da gestante selecionada, no segundo semestre de 2025, bem como no hospital escolhido por ela, o Hospital Filantrópico Policlínica ou o Hospital Filantrópico São Lucas, ambos localizados na cidade de Pato Branco, Paraná.

A intervenção ocorreu de forma individual, com a realização de três sessões de treinamento respiratório na casa da gestante durante um mês, cada uma com duração média de 40 minutos. Essas sessões tiveram como objetivo orientar a parturiente na prática de técnicas como respiração abdominal profunda, respiração torácica lenta e respiração de pressão, adaptadas conforme as fases do trabalho de parto. Além dos encontros presenciais, a participante foi incentivada a praticar os exercícios em casa, com apoio e monitoramento remoto via WhatsApp. O acompanhamento da aplicação das técnicas também aconteceu durante o trabalho de parto, em ambiente hospitalar, com supervisão da pesquisadora e de uma fisioterapeuta, garantindo a segurança da paciente.

A técnica aplicada neste estudo baseou-se no protocolo descrito por \citeonline{almeida2005}, o qual propõe estratégias de respiração e relaxamento como forma de favorecer a experiência do trabalho de parto. Embora este tenha sido o principal referencial metodológico adotado, destaca-se que o artigo utilizado também se apoia em autores clássicos, como Maldonado \citeyear{maldonado1991} e Lucas \citeyear{lucas1983}, que exploram os aspectos psicológicos da gestação, do parto e do puerpério, bem como a facilitação psicológica do parto. Esses referenciais ofereceram embasamento teórico complementar e reforçaram a importância das técnicas utilizadas na assistência à parturiente.

\section{Aspectos Éticos}
\label{sec:etica}

A pesquisa foi submetida à apreciação do Comitê de Ética em Pesquisa, em conformidade com a Resolução nº 466/2012 do Conselho Nacional de Saúde, que trata das diretrizes e normas regulamentadoras de pesquisas envolvendo seres humanos. A participante foi devidamente informada sobre os objetivos do estudo, e os riscos envolvidos foram considerados mínimos, relacionados à possibilidade de exposição indevida de dados pessoais. Esses riscos foram mitigados por meio de codificação, sigilo das informações e assinatura do Termo de Consentimento Livre e Esclarecido, bem como do termo de uso de imagem, assegurando o anonimato em registros audiovisuais.

Entre os benefícios observados, destacaram-se a contribuição para o fortalecimento da fisioterapia obstétrica, a valorização de sua atuação no contexto do parto humanizado e a importância das técnicas não farmacológicas, como a respiração controlada, na promoção do bem-estar da parturiente. Ademais, o estudo ressaltou os impactos positivos das práticas fisioterapêuticas na experiência do trabalho de parto, especialmente quando realizadas em um contexto acolhedor e respeitoso às necessidades físicas, emocionais e espirituais da mulher.

\chapter{Resultados}
\label{chap:resultados}

Este capítulo apresenta os dados obtidos a partir do acompanhamento fisioterapêutico da parturiente. Os resultados foram organizados em subseções que contemplam a caracterização da participante, a evolução da dor e as percepções qualitativas registradas durante o processo de parturição.

\section{Caracterização da Participante}
\label{sec:caracterizacao}

Descreve-se nesta seção o perfil sociodemográfico e obstétrico da participante, incluindo idade, contexto familiar, histórico gestacional e condições clínicas relevantes para a compreensão do caso acompanhado.

\section{Evolução da Dor e da Respiração}
\label{sec:evolucao}

Apresentam-se a seguir as medidas e narrativas referentes à evolução da dor e à aplicação das técnicas de respiração controlada em cada fase do trabalho de parto. Devem ser incluídos os instrumentos de avaliação utilizados, a frequência das contrações, as adaptações das técnicas e demais observações clínicas pertinentes.

\section{Percepções Qualitativas}
\label{sec:percepcoes}

Compilam-se aqui os relatos da parturiente, da doula e da equipe de fisioterapia acerca da experiência vivenciada, destacando aspectos emocionais, espirituais e funcionais atribuídos à intervenção.

\chapter{Discussão}
\label{chap:discussao}

Neste capítulo, interpretam-se os resultados obtidos à luz da literatura científica vigente. A análise crítica deve considerar convergências e divergências em relação a estudos prévios, bem como as especificidades do caso acompanhado.

\section{Integração com a Literatura}
\label{sec:integracao-literatura}

São relacionados os achados desta pesquisa com evidências nacionais e internacionais sobre técnicas de respiração controlada no trabalho de parto, destacando-se contribuições, limitações e implicações clínicas.

\section{Implicações para a Prática Fisioterapêutica}
\label{sec:implicacoes-pratica}

Apresentam-se as repercussões observadas para a atuação da fisioterapia obstétrica em contextos hospitalares e domiciliares, incluindo recomendações para protocolos assistenciais, formação profissional e ações interdisciplinares.

\chapter{Considerações Finais}
\label{chap:consideracoes-finais}

As considerações finais sintetizam os principais resultados, suas contribuições para a área da fisioterapia obstétrica e as perspectivas de continuidade. Devem ser destacadas as limitações encontradas, as sugestões para pesquisas futuras e os desdobramentos possíveis para a prática assistencial.

%---------- Referências ----------
\clearpage % garante a contagem correta de páginas para a ficha catalográfica
\phantomsection % cria um alvo de hiperlink consistente para o início das referências
\label{bibstart}
\bibliographystyle{tex/estilos/abnt-alf}
\bibliography{tex/bibliografia/abnt-options,tex/bibliografia/main}
\label{bibend}

\end{document}

\documentclass[12pt,oneside,a4paper]{article}

% -------------------------------------------------------
% Pacotes básicos e configuração de idioma
% -------------------------------------------------------
\usepackage[utf8]{inputenc}        % Codificação UTF-8
\usepackage[T1]{fontenc}           % Codificação de fonte
\usepackage[brazil]{babel}         % Idioma português do Brasil
\usepackage{times}                 % Fonte Times New Roman
\usepackage{setspace}              % Espaçamento entre linhas
\usepackage{geometry}              % Margens
\usepackage{color}                 % Cores (para texto vermelho, etc.)
\usepackage{hyperref}              % Links clicáveis no PDF
\usepackage{tex/estilos/abntcite}  % Citações no formato ABNT
\usepackage{graphicx}              % Inclusão de figuras
\usepackage{indentfirst}           % Recuo no primeiro parágrafo
\usepackage{xcolor}                % Cores adicionais

% -------------------------------------------------------
% Configurações de layout
% -------------------------------------------------------
\geometry{a4paper, left=3cm, right=2cm, top=3cm, bottom=2cm}
\setstretch{1.5}                   % Espaçamento 1,5
\hypersetup{
    colorlinks=true,
    linkcolor=black,
    citecolor=blue,
    urlcolor=blue
}

% -------------------------------------------------------
% Início do Documento
% -------------------------------------------------------
\begin{document}

\title{\textbf{Relatório de Avaliação Estrutural e Organizacional do Manuscrito}}
\author{Mateus Marochi Olenik}
\date{\today}

\maketitle

\tableofcontents

\section*{Análise Estrutural e Organizacional}
\addcontentsline{toc}{section}{Análise Estrutural e Organizacional}
A estrutura geral do manuscrito segue o modelo institucional, com capítulos dispostos em ordem lógica. Entretanto, o Capítulo de Resultados permanece sem dados empíricos, funcionando como orientação de preenchimento futuro, o que inviabiliza a avaliação científica do estudo e compromete a continuidade da Discussão e das Considerações Finais. Recomenda-se inserir imediatamente as informações coletadas, mantendo coerência com os objetivos específicos e garantindo a articulação entre métodos e achados (\textcolor{red}{Inserir descrição detalhada dos dados coletados, instrumentos utilizados, escalas de dor aplicadas e evolução temporal da paciente, relacionando-os a tabelas ou figuras conforme ABNT}).

\section*{Análise de Coesão e Coerência}
\addcontentsline{toc}{section}{Análise de Coesão e Coerência}
A Introdução apresenta boa contextualização do parto vaginal e da relevância da fisioterapia obstétrica, mas o último parágrafo introduz elementos da doutrina católica sem articulação direta com o problema científico, o que fragiliza a coesão temática. É importante justificar de modo explícito a relação entre espiritualidade e manejo fisioterapêutico da dor ou restringir essa discussão a seções pertinentes, como percepções qualitativas (\textcolor{red}{Reescrever indicando como a espiritualidade emergiu dos dados coletados ou transferir esse conteúdo para a seção de percepções, explicando sua pertinência clínica}). Ademais, o Problema de Pesquisa está redigido no pretérito perfeito, enquanto a formulação científica costuma empregar presente ou futuro do pretérito para expressar intencionalidade investigativa. Sugere-se adequar tempos verbais em consonância com o modelo (\textcolor{red}{Reformule para ``Investiga-se de que maneira...'' ou ``Busca-se compreender...''}).

\section*{Metodologia}
\addcontentsline{toc}{section}{Metodologia}
A seção metodológica delimita tipo de estudo, cenário, critérios de inclusão e exclusão, bem como etapas do protocolo. No entanto, faltam detalhes fundamentais sobre instrumentos de avaliação da dor (por exemplo, Escala Visual Analógica, Escala Numérica ou outros), procedimentos de registro das informações e estratégias de análise qualitativa. Sem esses elementos, não é possível validar a replicabilidade do relato de caso nem compreender o rigor científico empregado (\textcolor{red}{Adicionar subsubseção descrevendo instrumentos, periodicidade de coleta, responsáveis e critérios de análise das narrativas, incluindo referência metodológica específica}). Além disso, menciona-se acompanhamento remoto via WhatsApp, mas não se explicitam critérios éticos para guarda dos dados ou consentimento para comunicações digitais. Inclua justificativa e medidas de segurança adotadas.

Também se recomenda contextualizar a escolha das técnicas fisioterapêuticas com evidências atuais, destacando que intervenções como a bandagem neuromuscular e os protocolos de relaxamento guiado mostraram-se eficazes na redução da dor e do estresse durante o trabalho de parto em ensaios clínicos recentes \cite{miquelutti2018,kaple2023}. Além disso, é pertinente reforçar estratégias respiratórias e métodos não farmacológicos detalhados em revisões contemporâneas \cite{park2021,euzebioklein2022}, articulando-os com a aplicação clínica descrita no caso acompanhado.

\section*{Resultados e Discussão}
\addcontentsline{toc}{section}{Resultados e Discussão}
Os capítulos destinados aos resultados e à discussão encontram-se em formato de orientações, sem conteúdo analítico. Para garantir consistência, é imprescindível apresentar dados concretos (escores de dor, descrições das fases do trabalho de parto, falas selecionadas) e discuti-los frente à literatura citada. Recomenda-se estruturar subtópicos que retomem cada objetivo específico, evitando repetições e mantendo o diálogo com referências atualizadas (\textcolor{red}{Inserir tabela com evolução da dor, quadro com principais intervenções e análise comparativa com autores recentes}). Ao discutir, utilize tempos verbais no presente ao citar literatura consolidada e pretérito ao relatar achados do estudo, incluindo considerações sobre a dor pélvica no puerpério imediato e no seguimento, à luz de sínteses clínicas recentes \cite{hroncova2023}.

Além dos dados produzidos no caso, é essencial relacionar os achados às sínteses contemporâneas sobre recursos fisioterapêuticos, incluindo revisões e ensaios que analisam hidroterapia, exercícios com bola suíça e métodos manuais de analgesia, os quais reforçam benefícios maternos quando aplicados de forma estruturada \cite{melladogarcia2024,aslantas2023,smith2018}. Complementarmente, a incorporação de revisões e protocolos publicados em língua portuguesa fortalece o diálogo com a realidade assistencial brasileira, destacando experiências nacionais com hidroterapia, deambulação, bola suíça e orientações fisioterapêuticas integradas \cite{rocha2023,dantas2022,abreu2019,batista2023}.

\section*{Considerações Finais}
\addcontentsline{toc}{section}{Considerações Finais}
A seção final também está redigida como instrução. É necessário sintetizar resultados efetivos, destacar limitações (por exemplo, relato de caso único) e sugerir pesquisas futuras com delineamentos robustos. Aproveite para mencionar implicações práticas para a fisioterapia obstétrica e possíveis estratégias de implementação em serviços de saúde (\textcolor{red}{Elaborar parágrafo conclusivo com retomada do objetivo geral, principais benefícios observados, limitações e recomendações para estudos multicêntricos ou ensaios clínicos}).

\section*{Revisão Bibliográfica}
\addcontentsline{toc}{section}{Revisão Bibliográfica}
A revisão recorre a referências pertinentes, porém majoritariamente anteriores a 2010, excetuando-se \citeonline{raja2020} e \cite{sousa2021}. É aconselhável atualizar a base bibliográfica com estudos dos últimos cinco anos sobre respiração controlada e fisioterapia no parto, incluindo revisões sistemáticas e diretrizes clínicas (\textcolor{red}{Buscar bases como PEDro, Scielo e PubMed para incorporar evidências entre 2019 e 2024, priorizando artigos com metodologia robusta}). Verifique também a consistência entre citações no texto e registros no arquivo bibliográfico \texttt{main.bib}. Priorizem-se ainda sínteses em língua portuguesa que descrevem protocolos assistenciais voltados à realidade brasileira, ampliando o repertório de consultas rápidas para a equipe multiprofissional \cite{rocha2023,dantas2022,abreu2019,batista2023}.

Ao incorporar essas referências, priorize materiais com descrição detalhada dos protocolos terapêuticos e desfechos clínicos, como ensaios randomizados que avaliaram bandagens elásticas e exercícios em bola, bem como revisões de escopo e meta-sínteses sobre hidroterapia e métodos manuais \cite{miquelutti2018,aslantas2023,melladogarcia2024,smith2018}.

\section*{Conformidade com ABNT}
\addcontentsline{toc}{section}{Conformidade com ABNT}
O documento preserva elementos obrigatórios (capa, folha de rosto, dedicatória, resumo/abstract) e utiliza o pacote \texttt{abntcite}, alinhando-se às normas. Contudo, é preciso inserir listas de figuras e tabelas apenas quando houver elementos correspondentes e garantir que as seções futuras contenham títulos numerados coerentes com o sumário. Reforce a apresentação das palavras-chave em ordem alfabética e verifique a limitação de três a cinco termos, conforme NBR 6028.

% -------------------------------------------------------
\bibliographystyle{abnt}
\bibliography{main}

\end{document}
